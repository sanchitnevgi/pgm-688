%% LyX 2.2.2 created this file.  For more info, see http://www.lyx.org/.
%% Do not edit unless you really know what you are doing.
\documentclass[11pt,english]{article}
\usepackage[T1]{fontenc}
\usepackage[latin1]{inputenc}
\usepackage{amsmath}
\usepackage{amssymb}
\PassOptionsToPackage{normalem}{ulem}
\usepackage{ulem}

\makeatletter
%%%%%%%%%%%%%%%%%%%%%%%%%%%%%% User specified LaTeX commands.


\oddsidemargin=0in
\evensidemargin=0in
\textwidth=6.3in
\topmargin=-0.5in
\textheight=9in

\parindent=0in
%\pagestyle{empty}


\def\changemargin#1#2{\list{}{\rightmargin#2\leftmargin#1}\item[]}
\let\endchangemargin=\endlist 

%------------------------------------------------------------------
% PROBLEM, PART, AND POINT COUNTING...

% Create the problem number counter.  Initialize to zero.
\newcounter{problemnum}

% Specify that problems should be labeled with arabic numerals.
\renewcommand{\theproblemnum}{\arabic{problemnum}}


% Create the part-within-a-problem counter, "within" the problem counter.
% This counter resets to zero automatically every time the PROBLEMNUM counter
% is incremented.
\newcounter{partnum}[problemnum]

% Specify that parts should be labeled with lowercase letters.
\renewcommand{\thepartnum}{\arabic{partnum}}

% Make a counter to keep track of total points assigned to problems...
\newcounter{totalpoints}

% Make counters to keep track of points for parts...
\newcounter{curprobpts}		% Points assigned for the problem as a whole.
\newcounter{totalparts}		% Total points assigned to the various parts.

% Make a counter to keep track of the number of points on each page...
\newcounter{pagepoints}
% This counter is reset each time a page is printed.

% This "program" keeps track of how many points appear on each page, so that
% the total can be printed on the page itself.  Points are added to the total
% for a page when the PART (not the problem) they are assigned to is specified.
% When a problem without parts appears, the PAGEPOINTS are incremented directly
% from the problem as a whole (CURPROBPTS).


%---------------------------------------------------------------------------


% The \problem environment first checks the information about the previous
% problem.  If no parts appeared (or if they were all assigned zero points,
% then it increments TOTALPOINTS directly from CURPROBPTS, the points assigned
% to the last problem as a whole.  If the last problem did contain parts, it
% checks to make sure that their point values total up to the correct sum.
% It then puts the problem number on the page, along with the points assigned
% to it.

\newenvironment{problem}[1]{
% STATEMENTS TO BE EXECUTED WHEN A NEW PROBLEM IS BEGUN:
%
% Increment the problem number counter, and set the current \ref value to that
% number.
\refstepcounter{problemnum}
%
% Add some vspace to separate from the last problem.
\vspace{0.15in} \par
%
\setcounter{curprobpts}{#1} \setcounter{totalparts}{0}	% Reset counters.
%
% Now put in the "announcement" on the page.
\noindent{\Large \bf Question \theproblemnum. \normalsize ({\it \arabic{curprobpts} point\null\ifnum \value{curprobpts} = 1\else s\fi}\/)}
}{
% STATEMENTS TO BE EXECUTED WHEN AN OLD PROBLEM IS ENDED:
%
% If no parts to problem, then increment TOTALPOINTS and PAGEPOINTS for the
% entire problem at once.
\ifnum \value{totalparts} = 0
	\addtocounter{totalpoints}{\value{curprobpts}}	% Add pts to total.
	\addtocounter{pagepoints}{\value{curprobpts}}	% Add pts to page total.
%
% If there were parts for the problem, then check to make sure they total up
% to the same number of points that the problem is worth. Issue a warning
% if not.
\else \ifnum \value{totalparts} = \value{curprobpts}
	\else \typeout{}
	\typeout{!!!!!!!   POINT ACCOUNTING ERROR   !!!!!!!!}
	\typeout{PROBLEM [\theproblemnum] WAS ALLOCATED \arabic{curprobpts} POINTS,}
	\typeout{BUT CONTAINS PARTS TOTALLING \arabic{totalparts} POINTS!}
	\typeout{}
	\fi
\fi
}


%---------------------------------------------------------------------------


% The \newpart command increments the part counter and displays an appropriate
% lowercase letter to mark the part.  It adds points to the point counter
% immediately.  If 0 points are specified, no point announcement is made.
% Otherwise, the announcement is in scriptsize italics.

\newcommand{\newpart}[2]{
\refstepcounter{partnum}	% Set the current \ref value to the part number.
\begin{changemargin}{0in}{0in}
  \noindent\textbf{\theproblemnum.\thepartnum}~(\textit{#1 points}): 
  \textit{ #2}
\end{changemargin}
\addtocounter{totalparts}{#1}	% Add points to totalparts for this problem.
\addtocounter{pagepoints}{#1}	% Add points to total for this page.
\addtocounter{totalpoints}{#1}	% Add points to total for entire test.
}

\newcommand{\answerpart}[1]{
\begin{changemargin}{0.25in}{0in}
\noindent \textbf{Answer:} {#1}
\end{changemargin}	
}

\newcommand{\answer}[1]{
\textbf{Answer:} {#1}
}



%---------------------------------------------------------------------------



% Just in case you want to skip some numbers in your test...

\newcommand{\skipproblem}[1]{\addtocounter{problemnum}{#1}}



%---------------------------------------------------------------------------


% The \showpoints command simply gives a count of the total points read in up to
% the location at which the command is placed.  Typically, one places one
% \showpoints command at the end of the latex file, just prior to the
% \end{document} command.  It can appear elsewhere, however.

\newcommand{\showpoints}
{
\typeout{}
\typeout{====> A TOTAL OF \arabic{totalpoints} POINTS WERE READ.}
\typeout{}
}


%---------------------------------------------------------------------------



\usepackage[english]{babel}
\usepackage{times}
\usepackage{subfigure}


\newcommand{\argmax}{\mathop{\arg\max}}
\newcommand{\deriv}[1]{\frac{\partial}{\partial {#1}} }
\newcommand{\dsep}{\mbox{dsep}}
\newcommand{\Pa}{\mathop{Pa}}
\newcommand{\ND}{\mbox{ND}}
\newcommand{\De}{\mbox{De}}
\newcommand{\Ch}{\mbox{Ch}}
\newcommand{\graphG}{{\mathcal{G}}}
\newcommand{\graphH}{{\mathcal{H}}}
\newcommand{\setA}{\mathcal{A}}
\newcommand{\setB}{\mathcal{B}}
\newcommand{\setS}{\mathcal{S}}
\newcommand{\setV}{\mathcal{V}}
\DeclareMathOperator*{\union}{\bigcup}
\DeclareMathOperator*{\intersection}{\bigcap}
\DeclareMathOperator*{\Val}{Val}
\newcommand{\mbf}[1]{{\mathbf{#1}}}
\newcommand{\eq}{\!=\!}

\newcommand{\discrete}{stripes}
\newcommand{\gaussian}{swirl}

\newcommand{\update}[1]{\textcolor{blue}{[#1]}}


\usepackage{ascii}

\makeatother

\usepackage{babel}
\begin{document}
{\centering
	\rule{6.3in}{2pt}
	\vspace{1em}
	{\Large
		CS688: Graphical Models - Spring 2020\\
		Assignment 5\\
	}
	\vspace{1em}
	Assigned: Wed, Apr 21st. Due: Mon, May 1st, 17:00pm. \\
	\vspace{0.1em}
	\rule{6.3in}{1.5pt}
}\vspace{1em}

\textbf{General Instructions:} Submit a report with the answers to
each question at the start of class on the date the assignment is
due. You are encouraged to \emph{typeset your solutions}. To help
you get started, the full \LaTeX{}source of the assignment is included
with the assignment materials. For your assignment to be considered
``on time'', you must upload a zip file containing all of your code
to Moodle by the due date. Make sure the code is sufficiently well
documented that it's easy to tell what it's doing. You may use any
programming language you like. For this assignment, you \textbf{may}
\textbf{not} use existing code libraries for sampling or classification,
but you \textbf{may} (and are encouraged to) use a library for automatic
differentiation. If you think you've found a bug with the data or
an error in any of the assignment materials, please post a question
to the Piazza discussion forum. Make sure to list in your report any
outside references you consulted (books, articles, web pages, etc.)
and any students you collaborated with.\\

\textbf{Academic Honesty Statement:} Copying solutions from external
sources (books, web pages, etc.) or other students is considered cheating.
Sharing your solutions with other students is also considered cheating.
Any detected cheating will result in a grade of -100\% on the assignment
for all students involved, and potentially a grade of F in the course.\\

\textbf{Introduction:} In this assignment, you will experiment with
Bayesian inference and Stochastic Gradient Variational
Inference. Specifically, you will implement Bayesian logistic regression
with stochastic gradient variational inference.\\

\textbf{Logistic Regression:} Consider a logistic conditional distribution
over $y\in\{-1,+1\}$ given $x\in\mathbb{R}^{D}$ and a vector of
parameters $z\in\mathbb{R}^{D}$

\[
p(y|x,z)=\begin{cases}
\sigma\left(z^{T}x\right) & y=1\\
1-\sigma\left(z^{T}x\right) & y=-1
\end{cases}
\]
for $\sigma(s)=1/(1+\exp(-s)).$ Or, equivalently, $p(y|x,z)=\sigma(yz^{T}x).$\\
\\
\textbf{Prior Distribution}: For your prior on $z$ please use a standard
Normal distribution, with $p(z)\propto\exp\left(-\frac{1}{2}\Vert z\Vert^{2}\right)$\\

\textbf{Data}: You are given a dataset of 100 training inputs (\texttt{X\_train.csv})
and outputs (\texttt{Y\_train.csv}) and 1000 test inputs (\texttt{X\_train.csv})
and outputs (\texttt{Y\_train.csv}). Each vector $x$ has 5 dimensions.
(There is also data \texttt{X\_forextracreditonly.csv} and \texttt{Y\_forextracreditonly.csv}
but as the names imply these should be completely ignored unless you
are doing the extra credit problems.) \\

\begin{problem}{10} \textbf{Derivation of likelihood} Mathematically
derive an expression for $\log p(y|x,z)$. Write your answer using
the function $\text{lae}(s,t)=\log(\exp(s)+\exp(t)).$ Simplify your
answer as much as possible.
\end{problem} \\

\begin{problem}{10} \textbf{Derivation of gradient} First, mathematically
derive the derivative of $\text{lae}(s,t)$ with respect to $t$.
Next, derive the gradient of $\log p(y|x,z)$ with respect to $z$,
$\nabla\log p(y|x,z).$
\end{problem} \\

\begin{problem}{10} \textbf{Pseudocode for Stochastic gradient
variational inference (SGVI)} Use as a variational distribution a
Gaussian distribution with a mean of $w \in R^D$ and a fixed covariance
$\Sigma=0.5I$. Write pseudocode to approximately compute $\int_{z}q_{w}(z)\log p(z,\text{Data}),$
where the integral over $z$ is approximated by drawing a single samples
from $z$. Since the covariance is fixed, $q_w$ has a fixed entropy, and so this part
of the ELBO can be ignored. Also give pseudocode
to compute the gradient of your approximation with respect to $w$.
\end{problem} 

\begin{problem}{10} \textbf{Implement SGVI} Implement SGVI using
a fixed step size of $\epsilon=.005$ for a number of iterations $t_{\max}=10000$.
Make a plot of the weights evolving over time and show them as a plot
with five lines, one for each dimension of $w$.
\end{problem}

\begin{problem}{20} \textbf{Direct predictive accuracy for SGVI}
Given a set of samples $z_{1},...,z_{\text{t}_{\max}}$
from the posterior $p(z|\text{Data})$, give a mathematical equation
approximating the probability that $y=+1$ for a new test input $x$.
Write a function that takes as input a set of samples $z$ along with
a test set of inputs $x$ and computes the probability that each corresponding
output is $+1$.

Run SGVI five times for each number of iterations $t_{\max}\in\{10,100,1000,10000\}$.
After inference is complete, take the final value of $w$, and draw
$1000$ samples $z$ from the final variational
approximation $q_{w}(z)$ and use that set of $z$ to evaluate on test data\footnote{To be clear: SGVI does an optimization over $w$. But once we are
done, we take the final $w$ and draw a set of 1000 vectors $z$.
Then $w$ can be discarded, and these $z$ can be used to compute
predictions.}. Give a $4\times5$ table with all the error rates (four different time horizons and five
different repetitions). Make sure to label the axes of your table.
\end{problem}\\

\textbf{Extra Credit:} Finally, here are some extra-credit problems.
These are \emph{far} more difficult than the above problems and have
very small point values. These are also deliberately more open-ended,
leaving you more space for creativity. As a result, you will need
to carefully describe exactly what you did for each problem. To maximize
your score with limited time, you should make sure the above problems
are done thoroughly and ignore these. We will be very stingy in giving
credit for these problems-{}- do them only for the glory, and only
at your own risk!

\begin{problem}{5} \textbf{Step Sizes and Run Lengths for SGVI
Dynamics}

SGVI is based on stochastic gradient descent, and so finds
the optimum only in the $\epsilon\rightarrow0$ limit (even ignoring
local optima). However, a smaller step-size requires a longer run-length.
Systematically experiment with different step-sizes and different
run lengths. Make a graph of the average performance over time of
different step-sizes $\epsilon$ with time on the x-axes. Carefully
describe what you have done, and explain what your results seem to
suggest.
\end{problem}

\begin{problem}{5} \textbf{Minibatches for SGVI}
	
The above problems all involved computing $\nabla\log p(z|\text{Data})$ by summing the
likelihood over the full dataset. However, it is possible to instead
get an unbiased estimator of the gradient $\nabla\log p(z|\text{Data})$
by sampling from a small subset of the data. The \texttt{\{X,Y\}\_forextracreditonly.csv}
files contain a larger training set of 1000 points. For a fixed time
horizon of $t_{\max}=10000$ iterations, experiment with SGVI with
different minibatch sizes and different step sizes. Make a graph comparing
the test-set performance with several other choices of minibatch sizes
and step sizes. Make sure to compare to using all 1000 points in the
minibatch.
\end{problem}
\end{document}
